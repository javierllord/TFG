%%%%%%%%%%%%%%%%%%%%%%%%%%%%%%%%%%%%%%%%%%%%%%%%%%%%%%%%%%%%%%%%%%%%%%%%%%%%%%%!
%
% -- Resumen.tex --
%    Resumen
%
%%%%%%%%%%%%%%%%%%%%%%%%%%%%%%%%%%%%%%%%%%%%%%%%%%%%%%%%%%%%%%%%%%%%%%%%%%%%%%%!
%!TEX root = ../../Principal/TFG.tex
\chapter{Resumen}
\label{chap:resumen}
\pagestyle{fancy}
\thispagestyle{empty}
%
\graphicspath{{../Resumen/Imagenes/}}
\DeclareGraphicsExtensions{.pdf,.jpg,.png}
%
% Empieza a escribir aqu�

{\bf{Resumen}} 

En el mundo en el que vivimos, el reconocimiento biom�trico est� cada vez m�s presente en nuestras vidas; por ejemplo, hacemos uso de la autentificaci�n biom�trica cada vez que desbloqueamos nuestros smartphones con esc�neres de huellas dactilares o entramos en otros pa�ses mediante puertas de control de acceso fronterizo automatizadas con tecnolog�a de reconocimiento facial. En este proyecto se estudia un novedoso enfoque en este campo, que est� atrayendo cada vez m�s la atenci�n de la comunidad cient�fica: el uso del electrocardiograma (ECG) como caracter�stica biom�trica. Esta tecnolog�a todav�a inmadura para esta aplicaci�n, presenta grandes ventajas de cara al futuro, permitiendo la identificaci�n y autenticaci�n de personas a trav�s de la actividad el�ctrica del coraz�n, una se�al muy dif�cil de falsificar y que, adem�s, sirve como indicador de vida a la hora de detectar determinado tipo de ataques, en especial aquellos relacionados con la suplantaci�n del usuario.







En esta direcci�n, %para ello?
hemos dise�ado y desarrollado un sistema biom�trico a trav�s del ECG basado en el \textit{Deep Learning}, y m�s concretamente, haciendo uso de redes neuronales convolucionales. Para llevarlo a cabo, hemos utilizado una base de datos de 105 personas sanas, cuyas se�ales de ECG se han sometido a una fase de preprocesamiento para, a continuaci�n, realizar una extracci�n de sus caracter�sticas empleando una red neuronal convolucional y, finalmente, su clasificaci�n biom�trica. Este sistema se ha evaluado tanto en la modalidad de identificaci�n de usuarios como en la verificaci�n de la identidad, obteniendo resultados muy prometedores que nos permiten apoyar la afirmaci�n de que el ECG posee importantes caracter�sticas para el reconocimiento biom�trico. No obstante, es necesario seguir investigando en esta �rea tan novedosa y prometedora, para mejorar el rendimiento a largo plazo de estos sistemas biom�tricos basados en el ECG.


{\bf{Palabras clave}} 

ECG, electrocardiograma, biometr�a, identificaci�n, autenticaci�n, verificaci�n de la identidad, \textit{Deep Learning}, redes neuronales convolucionales.

\pagebreak[4]
\selectlanguage{british}

{\bf{Summary}} 

In the world we live nowadays, biometric recognition has become a key element in our lives; for example, we are authenticated with biometric traits every time we unlock our smartphones with our fingerprints or when we go through a facial recognition scan at the border checkpoints. In this project, we investigate a new approach in the field of biometric recognition that has captured the attention of the scientific community: the use of the electrocardiogram (ECG) as a biometric trait. This technology is still in a development phase, yet it offers great opportunities to identify and authenticate people by their heart electrical activity. Those signals are difficult to falsify and can be used to prove user's life to detect impersonation attacks.

In this line, we have designed and developed a biometric system that uses ECG based on Deep Learning, and more specifically based on Convolutional Neural Networks. We have used a database that contains data of 105 healthy people. The ECG of those people has been preprocessed to be later used in a feature extraction process by the CNN and finally been classified. This method has been used to identify users as well as to verify their identity. In both cases the results are promising and shows that the ECG can be used to perform biometric recognition. Nonetheless, it is still necessary to continue this line of research in order to improve the performance of these biometric systems based on the ECG.



%in the Convolutional Neural Network to extract their features and be able to classify them. 

{\bf{Keywords}} 

ECG, electrocardiogram, biometrics, identification, authentification, identity verification, Deep Learning, Convolutional Neural Networks.


\selectlanguage{spanish}


